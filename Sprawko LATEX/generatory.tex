\documentclass[a4paper,12pt]{article}
\usepackage[utf8]{inputenc}
\usepackage[MeX]{polski}
\usepackage{fixltx2e}
\usepackage[utf8]{inputenc}
\usepackage[T1]{fontenc}
\usepackage{graphicx}
\usepackage{color}
\usepackage{mathtools}
\usepackage{wasysym}
\usepackage[font=small,labelfont=bf]{caption}
\renewcommand*{\tablename}{Tab.}
\renewcommand*{\figurename}{Rys.}
%%%%%%%%%%%%%%%%%%%%%%%%%%%%%%%%%%%%%%%%%%%%%%%%%%%%%%%%%%%%%%%%%%%%%%%%%%%%%%%%%%%%%%%%%%%%%%%%%%%%%%%%%%%%%%%%%%%%%%%%%%%%%%%%%%%%%%%%%%%%%%%%%%%%
%%%%%%%%%%%%%%%%%%%%%%%%%%%%%%%%%%%%%%%%%%%%%%%%%%%%%%%%%%%%%STRONA TYTULOWA%%%%%%%%%%%%%%%%%%%%%%%%%%%%%%%%%%%%%%%%%%%%%%%%%%%%%%%%%%%%%%%%%%%%%%%%
%%%%%%%%%%%%%%%%%%%%%%%%%%%%%%%%%%%%%%%%%%%%%%%%%%%%%%%%%%%%%%%%%%%%%%%%%%%%%%%%%%%%%%%%%%%%%%%%%%%%%%%%%%%%%%%%%%%%%%%%%%%%%%%%%%%%%%%%%%%%%%%%%%%%
\title{\Huge \textbf{Politechnika Wrocławska\\[0.3in]} 
  \huge Katedra Teorii Pola, Układów elektronicznych i Optoelektronicznych \\[0.2in]
  \LARGE Zespół Układów Elektronicznych
}
\date{}
\author{}

\begin{document}
\maketitle

\begin{table}[h]
  \large
  \centering
  \begin{tabular}{|ll|l|}
    \hline
    \multicolumn{1}{|l|}{Data: 5.05.2015r}                & \multicolumn{2}{l|}{Dzień: Wtorek}                                     \\ \hline
    \multicolumn{1}{|l|}{Grupa: VII}                      & \multicolumn{2}{l|}{Godzina: 12:15-15:00}                              \\ \hline
    \multicolumn{3}{|l|}{\textit{\begin{tabular}[c]{@{}l@{}}\textbf{Temat ćwiczenia:} \\ Generatory kwarcowe\end{tabular}}}        \\ \hline
    \textbf{Dane projektowe:}                             & \multicolumn{2}{l|}{}                                                  \\ 
    Rezonator kwarcowy 8.000 MHz                          & \multicolumn{2}{l|}{}                                                                        \\ \hline
    \multicolumn{1}{|l|}{\textbf{l.p}}                    & \textbf{Nazwisko i imię}                 & \textbf{Oceny}              \\ \hline
    \multicolumn{1}{|l|}{1}                               & Arkadiusz Ziółkowski                     &                             \\ \hline
    \multicolumn{1}{|l|}{2}                               & Jakub Koban                              &                             \\ \hline
  \end{tabular}
\end{table}
%%%%%%%%%%%%%%%%%%%%%%%%%%%%%%%%%%%%%%%%%%%%%%%%%%%%%%%%%%%%%%%%%%%%%%%%%%%%%%%%%%%%%%%%%%%%%%%%%%%%%%%%%%%%%%%%%%%%%%%%%%%%%%%%%%%%%%%%%%%%%%%%%%%%
%%%%%%%%%%%%%%%%%%%%%%%%%%%%%%%%%%%%%%%%%%%%%%%%%%%%%%%%%%%%%%%%%%%ZADANIE PROJEKTOWE%%%%%%%%%%%%%%%%%%%%%%%%%%%%%%%%%%%%%%%%%%%%%%%%%%%%%%%%%%%%%%%
%%%%%%%%%%%%%%%%%%%%%%%%%%%%%%%%%%%%%%%%%%%%%%%%%%%%%%%%%%%%%%%%%%%%%%%%%%%%%%%%%%%%%%%%%%%%%%%%%%%%%%%%%%%%%%%%%%%%%%%%%%%%%%%%%%%%%%%%%%%%%%%%%%%%
\pagebreak
\section{Zadanie projektowe}
Zaprojektować:
\begin{enumerate}
\item {Generator kwarcowy Colpittsa-Pierce'a z tranzysotrem bipolarnym :}
  \begin{itemize}
  \item R\textsubscript{1} = 218.75 $k\Omega$
  \item R\textsubscript{2} = 0.9578 $k\Omega$
  \item C\textsubscript{1} = 33 pF
  \item C\textsubscript{2} = 3.3 nF
  \end{itemize}
\item {Generator kwarcowy zrealizowany na bramkach TTL :}
  \begin{itemize}
  \item R\textsubscript{5} = 560 $\Omega$
  \item R\textsubscript{6} = 1.8 $k\Omega$
  \item R\textsubscript{7} = 220 $\Omega$
  \item R\textsubscript{8} = 220 $k\Omega$
  \end{itemize}
\item{Generator kwarcowy realizowany na inwerterach CMOS :}
  \begin{itemize}
  \item R\textsubscript{9} = 10 $M\Omega$
  \item C\textsubscript{5} = 33 pF
  \item C\textsubscript{6} = 33 pF
  \end{itemize}
\item{Zasilanie :}
  \begin{itemize}
  \item C\textsubscript{7} = 33 pF
  \item C\textsubscript{8} = 33 pF
  \end{itemize}
\item Rezonator kwarcowy 8.000 MHz
\end{enumerate}
\pagebreak
%%%%%%%%%%%%%%%%%%%%%%%%%%%%%%%%%%%%%%%%%%%%%%%%%%%%%%%%%%%%%%%%%%%%%%%%%%%%%%%%%%%%%%%%%%%%%%%%%%%%%%%%%%%%%%%%%%%%%%%%%%%%%%%%%%%%%%%%%%%%%%%%%%%%
%%%%%%%%%%%%%%%%%%%%%%%%%%%%%%%%%%%%%%%%%%%%%%%%%%%%%%%%%%%%%%%%%%%CZĘŚĆ PROJEKTOWA%%%%%%%%%%%%%%%%%%%%%%%%%%%%%%%%%%%%%%%%%%%%%%%%%%%%%%%%%%%%%%%%%
%%%%%%%%%%%%%%%%%%%%%%%%%%%%%%%%%%%%%%%%%%%%%%%%%%%%%%%%%%%%%%%%%%%%%%%%%%%%%%%%%%%%%%%%%%%%%%%%%%%%%%%%%%%%%%%%%%%%%%%%%%%%%%%%%%%%%%%%%%%%%%%%%%%%
\section {Schematy projektowy}
%%%%%%%%%%%%%%%%%%%%%%%%%%%%%%%%%%%%%%%%%%%%%%%%%%%%%%%%%%%%%%%%%%%%%%%%%%%%%%%%%%%%%%%%%%%%%%%%%%%%%%%%%%%%%%%%%%%%%%%%%%%%%%%%%%%%%%%%%%%%%%%%%%%%
\begin{center}
  \includegraphics[width=0.5\textwidth]{u1}
  \captionof{figure}{Schemat generatora Colpittsa-Pierce'a }
\end{center}
%%%%%%%%%%%%%%%%%%%%%%%%%%%%%%%%%%%%%%%%%%%%%%%%%%%%%%%%%%%%%%%%%%%%%%%%%%%%%%%%%%%%%%%%%%%%%%%%%%%%%%%%%%%%%%%%%%%%%%%%%%%%%%%%%%%%%%%%%%%%%%%%%%%%
\begin{center}
  \includegraphics[width=0.65\textwidth]{u2} 
  \captionof{figure}{Schemat generatora na bramkach TTL  }
\end{center}
%%%%%%%%%%%%%%%%%%%%%%%%%%%%%%%%%%%%%%%%%%%%%%%%%%%%%%%%%%%%%%%%%%%%%%%%%%%%%%%%%%%%%%%%%%%%%%%%%%%%%%%%%%%%%%%%%%%%%%%%%%%%%%%%%%%%%%%%%%%%%%%%%%%%
\begin{center}
  \includegraphics[width=0.6\textwidth]{u3}
  \captionof{figure}{Schemat generatora na inwerterach CMOS  }
\end{center}
%%%%%%%%%%%%%%%%%%%%%%%%%%%%%%%%%%%%%%%%%%%%%%%%%%%%%%%%%%%%%%%%%%%%%%%%%%%%%%%%%%%%%%%%%%%%%%%%%%%%%%%%%%%%%%%%%%%%%%%%%%%%%%%%%%%%%%%%%%%%%%%%%%%%
%%%%%%%%%%%%%%%%%%%%%%%%%%%%%%%%%%%%%%%%%%%%%%%%%%%%%%%%%%%%%%%%%%%CZĘŚĆ LABORATORYJNA%%%%%%%%%%%%%%%%%%%%%%%%%%%%%%%%%%%%%%%%%%%%%%%%%%%%%%%%%%%%%%
%%%%%%%%%%%%%%%%%%%%%%%%%%%%%%%%%%%%%%%%%%%%%%%%%%%%%%%%%%%%%%%%%%%%%%%%%%%%%%%%%%%%%%%%%%%%%%%%%%%%%%%%%%%%%%%%%%%%%%%%%%%%%%%%%%%%%%%%%%%%%%%%%%%%
\section{Część laboratoryjna}\
%%%%%%%%%%%%%%%%%%%%%%%%%%%%%%%%%%%%%%%%%%%%%%%%%%%%%%%%%%%%%%%%%%%%%%%%%%%%%%%%%%%%%%%%%%%%%%%%%%%%%%%%%%%%%%%%%%%%%%%%%%%%%%%%%%%%%%%%%%%%%%%%%%%%
\subsection{Generator Colpittsa-Pierce'a }
\\ \\
\begin{center}
  \includegraphics[width=0.5\textwidth]{o3}
  \captionof{figure}{Przebieg generowanego sygnału}
\end{center}


\begin{table}[h]}
    \centering
    \captionof{table}{Wyniki pomiarów napięcia zasilania U oraz częstotliwośći f generowanego sygnału}
    \begin{tabular}{|c|c|c|} \hline
      \textbf{U {[}V{]}} & \textbf{f {[}MHz{]}} & \textbf{Odchylenie {[\permil]}} \\ \hline
      9         & 7.99934     & -0.082                 \\ \hline
      10        & 7.99935     & -0.081                 \\ \hline
      10.5      & 7.99935     & -0.081                 \\ \hline
      11        & 7.99936     & -0.080                 \\ \hline
      11.5      & 7.99936     & -0.080                 \\ \hline
      12        & 7.99937     & -0.079                 \\ \hline
      12.5      & 7.99937     & -0.079                 \\ \hline
      13        & 7.99938     & -0.077                 \\ \hline
      13.5      & 7.99938     & -0.077                 \\ \hline
      14        & 7.99939     & -0.076                 \\ \hline
      14.5      & 7.99939     & -0.076                 \\ \hline
      15        & 7.99940     & -0.075                 \\ \hline
    \end{tabular}
\end{table}

\begin{center}
  \includegraphics[width=1\textwidth]{z3}
  \captionof{figure}{Zależność odchylenia częstotliwości (od wartości nominalnej 8.000 MHz) od napięcia zasilania  }
\end{center}

Układ rozpoczął poprawną pracę od napięcia zasilania 9 V. Wraz ze wzrostem napięcia zmniejsza się odchylenie częstotliwości generowanego sygnału od wartościu zadanej (8.000 MHz). Maksymalne odchylenie od wartości nominalnej wynosi 0.082\permil. Generowany sygnał jest sinusoidalny.
%%%%%%%%%%%%%%%%%%%%%%%%%%%%%%%%%%%%%%%%%%%%%%%%%%%%%%%%%%%%%%%%%%%%%%%%%%%%%%%%%%%%%%%%%%%%%%%%%%%%%%%%%%%%%%%%%%%%%%%%%%%%%%%%%%%%%%%%%%%%%%%%%%%%
\subsection{Generator realizowany na bramkach TTL }
\\
\\
\begin{center}
  \includegraphics[width=0.5\textwidth]{o1} 
  \captionof{figure}{Przebieg generowanego sygnału}
\end{center}
\\
\begin{table}[h]
  \captionof{table}{Wyniki pomiarów napięcia zasilania U oraz częstotliwośći f generowanego sygnału }
  \centering
\begin{tabular}{|c|c|c|}
\hline
\textbf{U {[}V{]}} & \textbf{f {[}MHz{]}} & \textbf{Odchylenie {[\permil]}} \\ \hline
2.8                & 7.99827              & -0.216                     \\ \hline
3                  & 7.99861              & -0.174                     \\ \hline
3.3                & 7.99886              & -0.142                     \\ \hline
3.6                & 7.99896              & -0.130                     \\ \hline
3.9                & 7.99899              & -0.126                     \\ \hline
4.1                & 7.99899              & -0.126                     \\ \hline
4.3                & 7.99894              & -0.132                     \\ \hline
4.5                & 7.99896              & -0.130                     \\ \hline
4.7                & 7.99896              & -0.130                     \\ \hline
5                  & 7.99892              & -0.135                     \\ \hline
\end{tabular}
\end{table}

\begin{center}
  \includegraphics[width=1\textwidth]{z1}
  \captionof{figure}{Zależność odchylenia częstotliwości (od wartości nominalnej 8.000 MHz) od napięcia zasilania}
\end{center}
Układ rozpoczął poprawną pracę od napięcia zasilania 2.8 V. Wraz ze wzrostem napięcia zmniejsza się odchylenie częstotliwości generowanego sygnału od wartościu zadanej (8.000 MHz). Z godnie z charakterystyką na przedziale od ok.4 do 5 V układ wydaje się stabilizować generowany sygnał. Maksymalne odchylenie od wartości nominalnej wynosi 0.216\permil. Generowany sygnał przypomina prostokątny.

\pagebreak
%%%%%%%%%%%%%%%%%%%%%%%%%%%%%%%%%%%%%%%%%%%%%%%%%%%%%%%%%%%%%%%%%%%%%%%%%%%%%%%%%%%%%%%%%%%%%%%%%%%%%%%%%%%%%%%%%%%%%%%%%%%%%%%%%%%%%%%%%%%%%%%%%%%%
\subsection{Generator realizowany na inwerterach CMOS }
\begin{center}
  \center
  \includegraphics[width=0.5\textwidth]{o2}
  \captionof{figure}{Przebieg generowanego sygnału}
\end{center}

\begin{table}[h]
  \captionof{table}{Wyniki pomiarów napięcia zasilania U oraz częstotliwośći f generowanego sygnału}
  \centering
  \begin{tabular}{|c|c|c|}
\hline
\textbf{U {[}V{]}} & \textbf{f {[}MHz{]}} & \textbf{Odchylenie {[\permil]}} \\ \hline
5                  & 8.00035              & 0.044                      \\ \hline
5.5                & 8.00035              & 0.044                      \\ \hline
6                  & 8.00035              & 0.044                      \\ \hline
6.5                & 8.00035              & 0.044                      \\ \hline
7                  & 8.00035              & 0.044                      \\ \hline
7.5                & 8.00035              & 0.044                      \\ \hline
8                  & 8.00036              & 0.045                      \\ \hline
8.5                & 8.00036              & 0.045                      \\ \hline
9                  & 8.00036              & 0.045                      \\ \hline
9.5                & 8.00036              & 0.045                      \\ \hline
10                 & 8.00037              & 0.046                      \\ \hline
10.5               & 8.00038              & 0.047                      \\ \hline
11                 & 8.00039              & 0.049                      \\ \hline
11.5               & 8.00039              & 0.049                      \\ \hline
12                 & 8.00039              & 0.049                      \\ \hline
12.5               & 8.00040              & 0.050                      \\ \hline
13                 & 8.00040              & 0.050                      \\ \hline
13.5               & 8.00041              & 0.051                      \\ \hline
14                 & 8.00041              & 0.051                      \\ \hline
14.5               & 8.00042              & 0.053                      \\ \hline
15                 & 8.00042              & 0.053                      \\ \hline
\end{tabular}
\end{table} 

\begin{center}
  \includegraphics[width=1\textwidth]{z2}
  \captionof{figure}{Zależność odchylenia częstotliwości (od wartości nominalnej 8.000 MHz) od napięcia zasilania}
\end{center}
Układ rozpoczął poprawną pracę od napięcia zasilania 5 V. Wraz ze wzrostem napięcia zwiększa się odchylenie częstotliwości generowanego sygnału od wartościu zadanej (8.000 MHz). Maksymalne odchylenie od wartości nominalnej wynosi 0.053\permil. Generowany sygnał przypomina prostokątny.
\section {Wnioski}
\begin{enumerate}
\item Wszystkie generatory pozwalają na realizowanie swojej funkcji z bardzo dużą dokładnością. Najmniejszym odchyleniem od wartości zadanej (8.000 MHz) cechuje się generator zrealizowany w oparciu o inwertery CMOS (0.053\permil).
\item Generator Colpittsa-Pierce'a generuje sygnał sinusoidalny.
\item Sygnały z generatorów opartych na bramkach TTL oraz inwerterach CMOS są prostokątne z uwagi na cyfrowe bramki logiczne zastosowane do ich budowy.
\end{enumerate}
\end{document}
