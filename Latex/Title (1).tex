\documentclass[a4paper,12pt]{article}
\usepackage[utf8]{inputenc}
\usepackage[MeX]{polski}
\usepackage[table,xcdraw]{xcolor}
\usepackage[utf8]{inputenc}
\usepackage[T1]{fontenc}
\usepackage{graphicx}
\usepackage{color}	

\title{\Huge \textbf{Politechnika Wrocławska\\[0.3in]} 
\huge Katedra Teorii Pola, Układów elektronicznych i Optoelektronicznych \\[0.2in]
\LARGE Zespół Układów Elektronicznych
}
\date{}
\author{}

\begin{document}
\maketitle

\begin{table}[h]
\centering
\\[.9in]
\begin{tabular}{|l|l|l|l|}
\hline
\multicolumn{2}{|l|}{Data: 24.03.2015r.}                                                       & \multicolumn{2}{l|}{Dzień: Wtorek}                                                           \\ \hline
\multicolumn{2}{|l|}{Grupa: VII}                                                               & \multicolumn{2}{l|}{Godzina:12:15-15:00}                                                     \\ \hline
\multicolumn{4}{|l|}{{\begin{tabular}[c]{@{}l@{}}\textbf{Temat ćwiczenia:}\\ Liniowe zastosowania wzmacniaczy operacyjncyh\end{tabular}}}                                                       \\ \hline
\multicolumn{4}{|l|}{\begin{tabular}[c]{@{}l@{}}Dane Projektowe: \\ 
U_{we_{p-p}}=4V \, \, \,\;\;\quad\quad\quad\quad\quad\quad R1=820\Omega \\

U_{wy_{p-p}}=3V\, \, \,\;\quad\quad\quad\quad\quad\quad R2=9.1k\Omega \\
f_{} = 3 kHz\quad\quad\quad\quad\quad\quad\quad\quad R3=1k\Omega \\
\,\quad\quad\quad\quad\quad\quad\quad\quad\quad\quad\quad\quad C=6.8nF
\end{tabular}} \\ \hline
\textbf{l.p} & \multicolumn{2}{l|}{\textbf{Nazwisko i imię}} & \textbf{Oceny} \\ \hline
1 & \multicolumn{2}{l|}{Arkadiusz Źiółkowski} &  \\ \hline
2 & \multicolumn{2}{l|}{Jakub Koban} &  \\ \hline
\end{tabular}
\end{table}


\end{document}
