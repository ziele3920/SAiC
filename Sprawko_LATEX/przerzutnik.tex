\documentclass[a4paper,12pt]{article}
\usepackage[utf8]{inputenc}
\usepackage[MeX]{polski}
\usepackage{fixltx2e}
\usepackage[table,xcdraw]{xcolor}
\usepackage[utf8]{inputenc}
\usepackage[T1]{fontenc}
\usepackage{graphicx}
\usepackage{color}
\usepackage{mathtools}
%%%%%%%%%%%%%%%%%%%%%%%%%%%%%%%%%%%%%%%%%%%%%%%%%%%%%%%%%%%%%STRONA TYTULOWA%%%%%%%%%%%%%%%%%%%%%%%%%%%%%%%%%%%%%%%%%%%%%%%%%%%%%%%%%%%%%%%%%%%%%%%%
\title{\Huge \textbf{Politechnika Wrocławska\\[0.3in]} 
  \huge Katedra Teorii Pola, Układów elektronicznych i Optoelektronicznych \\[0.2in]
  \LARGE Zespół Układów Elektronicznych
}
\date{}
\author{}

\begin{document}
\maketitle
\begin{table}[h]
  \large
  \centering
  \begin{tabular}{|ll|l|}
    \hline
    \multicolumn{1}{|l|}{Data: 7.04.2015r}                & \multicolumn{2}{l|}{Dzień: Wtorek}                                     \\ \hline
    \multicolumn{1}{|l|}{Grupa: VII}                      & \multicolumn{2}{l|}{Godzina: 12:15-15:00}                               \\ \hline
    \multicolumn{3}{|l|}{\textit{\begin{tabular}[c]{@{}l@{}}\textbf{Temat ćwiczenia:} \\ Liniowe stabilizatory napięcia\end{tabular}}} \\ \hline
    \textbf{Dane projektowe:}                             & \multicolumn{2}{l|}{}                                                   \\
    T=0.35 \mu s                                          & \multicolumn{2}{l|}{}                                                    \\
    C=X \mu s                                          & \multicolumn{2}{l|}{}                                              \\
    R\textsubscript{a}=Y k\Omega}                          & \multicolumn{2}{l|}{}                                                      \\ \hline
    \multicolumn{1}{|l|}{\textbf{l.p}}                    & \textbf{Nazwisko i imię}                 & \textbf{Oceny}               \\ \hline
    \multicolumn{1}{|l|}{1}                               & Arkadiusz Ziółkowski                     &                               \\ \hline
    \multicolumn{1}{|l|}{2}                               & Jakub Koban                              &                                \\ \hline
  \end{tabular}
\end{table}


%%%%%%%%%%%%%%%%%%%%%%%%%%%%%%%%%%%%%%%%%%%%%%%%%%%%%%%%%%%%%%%%%%%%%%%%%%%%%%%%%%%%%%%%%%%%%%%%%%%%%%%%%%%%%%%%%%%%%%%%%%%%%%%%%%%%%%%%%%%%%%%%%%%%
%%%%%%%%%%%%%%%%%%%%%%%%%%%%%%%%%%%%%%%%%%%%%%%%%%%%%%%%%%%%%%%%%%%ZADANIE PROJEKTOWE%%%%%%%%%%%%%%%%%%%%%%%%%%%%%%%%%%%%%%%%%%%%%%%%%%%%%%%%%%%%%%%
%%%%%%%%%%%%%%%%%%%%%%%%%%%%%%%%%%%%%%%%%%%%%%%%%%%%%%%%%%%%%%%%%%%%%%%%%%%%%%%%%%%%%%%%%%%%%%%%%%%%%%%%%%%%%%%%%%%%%%%%%%%%%%%%%%%%%%%%%%%%%%%%%%%%
\pagebreak
\section{Zadanie projektowe}
Zaprojektować przerzutnik monostabilny w oparciu o układ scalony ``555'' dla T=35 \mu s
%%%%%%%%%%%%%%%%%%%%%%%%%%%%%%%%%%%%%%%%%%%%%%%%%%%%%%%%%%%%%%%%%%%%%%%%%%%%%%%%%%%%%%%%%%%%%%%%%%%%%%%%%%%%%%%%%%%%%%%%%%%%%%%%%%%%%%%%%%%%%%%%%%%%
%%%%%%%%%%%%%%%%%%%%%%%%%%%%%%%%%%%%%%%%%%%%%%%%%%%%%%%%%%%%%%%%%%%CZĘŚĆ PROJEKTOWA%%%%%%%%%%%%%%%%%%%%%%%%%%%%%%%%%%%%%%%%%%%%%%%%%%%%%%%%%%%%%%%%%
%%%%%%%%%%%%%%%%%%%%%%%%%%%%%%%%%%%%%%%%%%%%%%%%%%%%%%%%%%%%%%%%%%%%%%%%%%%%%%%%%%%%%%%%%%%%%%%%%%%%%%%%%%%%%%%%%%%%%%%%%%%%%%%%%%%%%%%%%%%%%%%%%%%%
\section{Obliczenia projektowe}

\begin{equation}
  T=R_A\cdot C\cdot ln\left(\frac{V_{CC}}{V_{CC}-\frac{2}{3}V_{CC}}\right)\approx 1.1\cdot R_A\cdot C=1.1 
\end{equation}

%%%%%%%%%%%%%%%%%%%%%%%%%%%%%%%%%%%%%%%%%%%%%%%%%%%%%%%%%%%%%%%%%%%%%%%%%%%%%%%%%%%%%%%%%%%%%%%%%%%%%%%%%%%%%%%%%%%%%%%%%%%%%%%%%%%%%%%%%%%%%%%%%%%%
\section {Schemat projektowy}
\begin{figure}[h]
  \center
  \includegraphics[width=1\textwidth]{schemat_ukladu}
  \caption{Schemat projektowanego układu}
\end{figure}
\pagebreak
%%%%%%%%%%%%%%%%%%%%%%%%%%%%%%%%%%%%%%%%%%%%%%%%%%%%%%%%%%%%%%%%%%%%%%%%%%%%%%%%%%%%%%%%%%%%%%%%%%%%%%%%%%%%%%%%%%%%%%%%%%%%%%%%%%%%%%%%%%%%%%%%%%%%
%%%%%%%%%%%%%%%%%%%%%%%%%%%%%%%%%%%%%%%%%%%%%%%%%%%%%%%%%%%%%%%%%%%CZĘŚĆ LABORATORYJNA%%%%%%%%%%%%%%%%%%%%%%%%%%%%%%%%%%%%%%%%%%%%%%%%%%%%%%%%%%%%%%
%%%%%%%%%%%%%%%%%%%%%%%%%%%%%%%%%%%%%%%%%%%%%%%%%%%%%%%%%%%%%%%%%%%%%%%%%%%%%%%%%%%%%%%%%%%%%%%%%%%%%%%%%%%%%%%%%%%%%%%%%%%%%%%%%%%%%%%%%%%%%%%%%%%%
\section{Część laboratoryjna}


%%%%%%%%%%%%%%%%%%%%%%%%%%%%%%%%%%%%%%%%%%%%%%%%%%%%%%%%%%%%%%%%%%%%%%%%%%%%%%%%%%%%%%%%%%%%%%%%%%%%%%%%%%%%%%%%%%%%%%%%%%%%%%%%%%%%%%%%%%%%%%%%%%%%

\section {Wnioski}
\begin{enumerate}  
\item 
\item
\item 
\end{enumerate}



\end{document}
